\documentclass[12pt,a4paper,twoside,openright]{report}
\usepackage[utf8]{inputenc}
\usepackage[T1]{fontenc}
\usepackage{tgtermes}
\usepackage{ragged2e} 
\usepackage[french]{babel}
\usepackage{microtype}
%\usepackage{kantlipsum}
\PassOptionsToPackage{hyphens}{url}
\usepackage{hyperref}
\usepackage{epigraph}
\usepackage{eso-pic,graphicx}
\usepackage{amsmath}
\usepackage{multirow}
\usepackage{booktabs}
\usepackage{longtable}
\usepackage{lscape}
\usepackage{appendix}
\usepackage[table,xcdraw]{xcolor, soul}
\usepackage{titletoc}
\sethlcolor{red}
\usepackage{fancyhdr}
\usepackage{float}
\usepackage{shorttoc}
\usepackage{lipsum}

%Pour gérer un problème de caractère dans la bibliographie
\usepackage{newunicodechar}
\newunicodechar{ᵉ}{\textsuperscript{e}}

%Pour forcer un retour à la ligne au milieu d'un mot
\newcommand\hyp{\penalty0\hskip0pt\relax}

%Citations en italiques avec guillemets français
\usepackage[babel=true]{csquotes}
\let\oldenquote\enquote
\renewcommand{\enquote}[1]{\oldenquote{\textit{#1}}}
\newenvironment{itblock}{\begin{quote}\itshape}{\end{quote}}
\SetBlockEnvironment{itblock}
\renewcommand{\mkcitation}[1]{\unskip\ \textnormal{(#1)}}


%Pour les infobulles
%Voir https://www.latex-fr.net/3_composition/texte/paragraphes/encadrer_du_texte
\usepackage{awesomebox}
\usepackage[most]{tcolorbox}
\newcommand{\githubcode}[2]{%
    \begin{awesomebox}[purple]{3pt}{\faCode[regular]}{purple}{\small \textbf{#1}
    
    \scriptsize \url{#2}}
    \end{awesomebox}
}


\tcbset{before upper={\parindent=20pt}} % définit un alinéa globalement pour toutes les boîtes tcolorbox

%Pour les sources des images et tableaux
\usepackage{copyrightbox}
\makeatletter
\renewcommand{\CRB@setcopyrightfont}{%
    \justifying
    \usefont{T1}{tgtermes}{n}{n}
    \footnotesize
    \color{black!66}
    \selectfont
    }
\makeatother


%Gestion du format de page
\usepackage[
  a4paper,
  twoside,
  %openright,
  inner=2.5cm,
  outer=2cm,
  top=1.5cm,
  bottom=1.5cm,
  includeheadfoot,
  headheight=10pt,
  headsep=20pt,
  footskip=30pt
]{geometry}
%Surface occupée par le texte : 165 mm × 245 mm

\setlength {\marginparwidth }{2cm}

%Interdire les lignes orphelines
\widowpenalty = 99999

%interligne
\renewcommand{\baselinestretch}{1.15}

%Gestion des interlignes entre paragraphes et indentation*
\usepackage[skip=6pt plus1pt, indent=20pt]{parskip}

%Gestion des notes de relecture (\todoLCE)
\usepackage[colorinlistoftodos,prependcaption]{todonotes}
\newcommand{\todoTL}[1]{\todo[color=green!40, author=Thibault, inline]{#1}}
\newcommand{\todoDJ}[1]{\todo[color=blue!40, author=Didier, inline]{#1}}
\newcommand{\todoLCE}[1]{\todo[color=yellow!40, author=Laure, inline]{#1}}

\usepackage{enumitem} 

%formatage de la bibliographie
\usepackage[
    backend=biber
    ,useprefix=false
    %,style=chicago-authordate
    ,style=authoryear-icomp
    ,sorting=nyt,
    %,giveninits=true,
    %uniquename=init,
    ,maxcitenames=2
    ,maxbibnames=9
    %,url=false
    %sortcites,
    %backref=false
]{biblatex}

\makeatletter
\AtBeginDocument{\toggletrue{blx@useprefix}}
\AtBeginBibliography{\togglefalse{blx@useprefix}}
\makeatother

%nettoyage de la bibliographie
\AtEveryBibitem{%
  \clearfield{note}%
  \clearfield{month}%
  \clearlist{language}%
  \clearfield{isbn}%
  \clearfield{issn}%
  %\clearfield{url}%
  %\DeclareFieldFormat{urldate}{}
}

\renewbibmacro*{doi+eprint+url}{%
  \begingroup
  \footnotesize
  \printfield{doi}%
  \newunit\newblock
  \printfield{eprint}%
  \newunit\newblock
  \ifboolexpr{
    test {\ifentrytype{online}}
    or test {\ifentrytype{software}}
    or test {\ifentrytype{report}}
    or test {\ifentrytype{misc}}
  }
  {%
    \printfield{url}%
    \setunit*{\addspace}%
    \printurldate
  }
  {}%
  \endgroup
}




\addbibresource{references.bib}
%\addbibresource{ref2.bib}


%On numérote les sections jusqu'à 2 (Partie + Chapitre + Section + sous-section)
\setcounter{secnumdepth}{2}
\titlecontents{chapter}[0em] {\addvspace{0.5em}\bfseries}{Chapitre~\thecontentslabel \hspace{1em}}{}{\hfill\contentspage}

\newcommand{\annexetoc}{
  \titlecontents{chapter}[0em]
    {\addvspace{0.5em}\bfseries}
    {Annexe~\thecontentslabel\hspace{1em}}
    {}
    {}
}


%Pastilles
\usepackage{tikz}
\definecolor{COM_GEN}{HTML}{08519c}%
\definecolor{PUB}{HTML}{8dd3c7}%
\newcommand{\pastille}[1]{%
  \tikz[baseline=-0.5ex]{
    \node[circle, fill=#1, inner sep=4pt, outer sep=0] {};
  }%
}


% Définition du point médian : $\cdot$
\def\cdt{\kern-0.5pt\ensuremath\cdot\kern-0.5pt}

% Diminue la fréquence de l'hyphenation
\microtypesetup{activate=true}


\renewcommand{\thefigure}{\thechapter.\arabic{figure}}


%pas d'entete et pied de page en début de chapitre
\fancypagestyle{plain}{%
  \fancyhf{}%
  \renewcommand{\headrulewidth}{0pt}%
  \fancyfoot[LE,RO]{\footnotesize \thepage}%
}


%Formatage des entêtes
\makeatletter
\renewcommand{\sectionmark}[1]{%
  \MakeUppercase{\markright{\footnotesize \thesection\quad #1}}%
}
\makeatother


%subsubsection en italique dans la table des matières
\makeatletter
\renewcommand{\l@subsubsection}[2]{%
  \ifnum \c@tocdepth >2 % vérifie qu'on affiche les subsubsections
    \addpenalty{\@secpenalty}%
    \vskip 1.0ex plus 0.2ex minus 0.2ex % espace vertical
    \setlength\@tempdima{3.8em} % indentation (ajuster si besoin)
    \begingroup
      \parindent \z@ \rightskip \@pnumwidth
      \parfillskip -\@pnumwidth
      \leavevmode \itshape % italique ici
      \advance\leftskip\@tempdima
      \hskip -\leftskip
      #1\nobreak\hfil \nobreak\hb@xt@\@pnumwidth{\hss #2}\par
    \endgroup
  \fi}
\makeatother

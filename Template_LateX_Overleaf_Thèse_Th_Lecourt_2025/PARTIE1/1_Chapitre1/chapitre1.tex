\graphicspath{PARTIE1/1_Chapitre1/images/}

Le foncier constitue un objet d'étude pluridisciplinaire et fait l'objet d'un regain d'intérêt récent dans la littérature scientifique.

Le point médian pour l'écriture inclusive de tou$\cdot$tes s'utilise de cette manière.


\section{Le foncier, matière première de l’aménagement}

Le foncier est souvent assimilé au sol. Selon le Trésor de la Langue Française, le terme n’est d’ailleurs qu’un adjectif qualifiant ce qui a trait aux biens-fonds.


\subsection{Le foncier, une ressource territoriale}

La notion de foncier renvoie généralement à trois conceptions, non exclusives les unes des autres.

%Exemple de citation en bloc
Cet objectif était aussi déjà formulé dans la Loi d'Orientation Foncière, trente ans plus tôt~:
    \blockquote[Loi d'Orientation Foncière, 1967, Titre I~--~Chapitre II~--~Art. 12]{Les schémas directeurs d’aménagement et d’urbanisme fixent les orientations fondamentales de l’aménagement des territoires intéressés, notamment en ce qui concerne l’extension des agglomérations. Compte tenu des relations entre ces agglomérations et les régions avoisinantes, et de l’équilibre qu’il convient de préserver entre l’extension urbaine, l’exercice d’activités agricoles [\ldots] et la conservation des massifs boisés et des sites naturels, ces schémas directeurs déterminent, en particulier, la destination générale des sols}.


\subsubsection{Le foncier, une délimitation juridique du sol pour y projeter des activités sociale}

La notion de ressource territoriale renvoie à une littérature scientifique s’intéressant aux objets
mobilisés par les acteurs du territoire pour son développement : pétrole, forêts, etc

%Exemple de box
\tipbox{\setlength{\parindent}{20pt}
    \noindent \textbf{Rappel~: les gisements fonciers libres}
    Les gisements fonciers libres sont les espaces cadastrés auxquels on a soustrait les principales contraintes techniques à leur mobilisation dans des opérations d'aménagements (emprises bâties, zones hydrographiques, pentes fortes, espaces pollués, etc.), ainsi que des contraintes d'usage ou de fonction (espaces dédiés à des usages publics, commerciaux, industriels, etc.).
}
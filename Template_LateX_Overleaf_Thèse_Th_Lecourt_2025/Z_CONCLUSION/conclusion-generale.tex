L'objectif de cette recherche était de lever le voile sur la propriété foncière communale en France hexagonale afin d'en analyser les composantes, les logiques et les dynamiques, à partir des données cadastrales. Cet objectif était justifié par le fait que les communes jouent un rôle de premier ordre dans la définition des orientations d'aménagement du territoire et la préservation d'un certain intérêt général local, et que leur propriété foncière constitue un levier essentiel à ces objectifs. Dans un contexte austéritaire, néolibéral et de privatisation du foncier public dans plusieurs pays du monde, mesurer le phénomène de privatisation du foncier communal constituait donc un enjeu crucial. 



\enquote{Aujourd’hui je reconnais qu’il n’y a plus de foncier, on a tellement mobilisé}, déclare Jean-Yves Mano, adjoint à la Maire de Paris en charge du logement de 2001 à 2014, au journal Mediapart. Le journal veut alerter sur \enquote{les fausses bonnes affaires de la mairie de Paris} (\textit{ibid.}) que représentent les ventes du foncier municipal\footnote{Les adjectifs <<~communal~>> et <<~municipal~>> sont considérés, dans cette thèse, comme parfaitement synonymes~: le premier terme puise son origine dans les communautés d’habitant$\cdot$es organisé$\cdot$s politiquement autour d’un serment commun au XI\ieme{} siècle~--~la \textit{communio}~--~; le second dans le \textit{municipium} de l'Antiquité romaine.} aux acteurs privés et en souligne tout le paradoxe. Si ces privatisations permettent effectivement de répondre à un besoin social et à un objectif politique~--~en l'occurrence, la construction de logements et notamment de logements sociaux~--, si elles permettent de générer des recettes d'investissement, une telle politique contient paradoxalement la négation d'elle-même~: elle prend fin quand la ressource foncière est épuisée. Cette stratégie à court-terme entrave la capacité à long-terme de la commune à aménager le territoire et réguler les marchés par la mobilisation de son foncier.

Toutefois, derrière le caractère symbolique de la situation particulière de la capitale française, la manière dont toutes les autres communes font face à ce dilemme d'aménagement ne fait pas l'objet du même traitement médiatique. Pourtant, plusieurs éléments de contexte incitent à penser que la vente de leur foncier constitue plus largement la voie privilégiée.


\subsubsection{Un dilemme d'aménagement favorable à la privatisation du foncier communal~?}

L'austérité budgétaire est une constante qui s'impose aux communes depuis une cinquantaine d'années, réaffirmée par les dernières propositions de budget par le gouvernement n'envisageant pas d'alternative à de nouvelles réductions des ressources allouées aux collectivités locales\footnote{Le gouvernement prévoyait une coupe budgétaire de 5 milliards d'euros dans les recettes des collectivités locales en 2025~: \url{https://www.lemonde.fr/politique/article/2024/10/08/les-elus-locaux-refusent-l-effort-de-5-milliards-d-euros-que-leur-demande-le-gouvernement_6346729_823448.html}}. La vente du foncier peut dès lors représenter une opportunité pour générer des recettes importantes et rapides, d'autant plus que les prix fonciers sont particulièrement hauts et continuellement à la hausse depuis les années 2000, constituant une aubaine pour la valorisation financière des réserves foncières municipales.

...

Enfin, la littérature scientifique internationale documente un phénomène de privatisation, parfois massive, du foncier public. L'exemple le plus emblématique et le plus proche de la France correspond sans doute au Royaume-Uni, où deux millions d'hectares, soit 10~\% du territoire, ont été vendus à des propriétaires privés depuis le début de l'ère Thatcher à la fin des années 1970, dont la moitié par les gouvernements locaux \parencite{christophers_new_2018}. Plusieurs cas d'étude ont aussi analysé des privatisations du foncier public d'État, affecté à des usages militaires et ferroviaires, notamment au Royaume-Uni \parencite{artioli_sale_2021}, en Italie et en France \parencite{adisson_four_2020}.

...
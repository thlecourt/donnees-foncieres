\graphicspath{PARTIE1/2_Chapitre2/images/}

Puisque la propriété foncière est un instrument important à la disposition des communes pour planifier l'aménagement, pourquoi n'est-elle pas généralisée et systématique~? Au contraire même, plusieurs indices suggèrent l'hypothèse que le foncier des communes serait en voie de réduction, dans un contexte de privatisation observé dans le monde et notamment en Europe. Comment ce contexte affecte-t-il spécifiquement la propriété foncière communale en France~? Quelles en sont les implications sur la capacité des communes à élaborer des politiques foncières et à maîtriser les opérations d'aménagement conformément à leur projet de territoire, dans l'intérêt général local~?


% Exemple d'image
\begin{figure}[!h]
    \label{fig:evol-ressources}
    \caption{Graphique d'évolution du prix des terrains à bâtir et des recettes communales}
    %\medskip
    \copyrightbox[b]{
        \centering
        \includegraphics[width=1\linewidth]{PARTIE1//2_Chapitre2//images/evol_prix_ressources.png}
        }
    {\justifying
    Réalisation~: Th. Lecourt, 2025. Sources~: DV3F v7 2022 (DGFiP-DGALN-CEREMA), Comptes consolidés des communes 2012-2021 (OFGL).\\
    
    NB~: les statistiques sont calculées à l'échelle de la France hexagonale. Les montants sont rapportés en base 100 afin de rendre comparables des ordres de grandeur différents. Les terrains à bâtir sont identifiés selon la catégorisation du CEREMA. Les transactions d'un montant inférieur à 1€ n'ont pas été prises en compte dans le calcul du prix médian au m². La version de DV3F utilisée datant de 2022, toutes les transactions réalisées en 2021 n'avaient pas été encore remontées aux services fiscaux~: le calcul du prix médian de 2021 pourrait s'en trouver affecté.}
\end{figure}


%Exemple d'encadré
\begin{tcolorbox}[title=Clés de lecture des graphiques de résultat des régressions logistiques multinomiales, colback=white!3!white, colframe=black!75!black, enhanced, breakable]
\label{sec:encadre-logit}
    Comme évoqué dans l'encadré dédié à la régression logistique du Chapitre 5, les graphiques de résultats du modèle représentent les \textit{odds-ratio} (OR) sur l'axe des abscisses. Les OR correspondent à la probabilité qu'une modalité soit observée plutôt que la modalité de référence, représentée par l'axe vertical noir. Par exemple, on peut dire que, dans l'aire d'attraction de Paris (\textit{com\_typeaav}), il est deux fois plus probable d'observer des intermédiations que des portages longs.
\end{tcolorbox}
%Introduction de la partie 1
\thispagestyle{plain}

\chapter*{Introduction de la première partie}
\markboth{INTRODUCTION DE LA PREMIÈRE PARTIE}{}
\thispagestyle{plain}
\label{chap:intro-p1}
%Définition des entêtes
\pagestyle{fancy}
\fancyhead{}
\fancyhead[R]{\footnotesize \rightmark}
\fancyfoot{}
\fancyfoot[LE,RO]{\footnotesize \thepage}
\fancyfoot[RE]{\footnotesize PARTIE \thepart}
\fancyfoot[LO]{\footnotesize Chapitre \thechapter}
%


L'objectif de cette première partie est d'appréhender la propriété foncière communale afin d'en souligner les différentes dimensions et les enjeux d'analyse.



\chapter{De la propriété à la maîtrise du foncier, un rôle central des communes dans l’aménagement}
\thispagestyle{plain}
\label{chap:1}
\graphicspath{PARTIE1/1_Chapitre1/images/}

Le foncier constitue un objet d'étude pluridisciplinaire et fait l'objet d'un regain d'intérêt récent dans la littérature scientifique.

Le point médian pour l'écriture inclusive de tou$\cdot$tes s'utilise de cette manière.


\section{Le foncier, matière première de l’aménagement}

Le foncier est souvent assimilé au sol. Selon le Trésor de la Langue Française, le terme n’est d’ailleurs qu’un adjectif qualifiant ce qui a trait aux biens-fonds.


\subsection{Le foncier, une ressource territoriale}

La notion de foncier renvoie généralement à trois conceptions, non exclusives les unes des autres.

%Exemple de citation en bloc
Cet objectif était aussi déjà formulé dans la Loi d'Orientation Foncière, trente ans plus tôt~:
    \blockquote[Loi d'Orientation Foncière, 1967, Titre I~--~Chapitre II~--~Art. 12]{Les schémas directeurs d’aménagement et d’urbanisme fixent les orientations fondamentales de l’aménagement des territoires intéressés, notamment en ce qui concerne l’extension des agglomérations. Compte tenu des relations entre ces agglomérations et les régions avoisinantes, et de l’équilibre qu’il convient de préserver entre l’extension urbaine, l’exercice d’activités agricoles [\ldots] et la conservation des massifs boisés et des sites naturels, ces schémas directeurs déterminent, en particulier, la destination générale des sols}.


\subsubsection{Le foncier, une délimitation juridique du sol pour y projeter des activités sociale}

La notion de ressource territoriale renvoie à une littérature scientifique s’intéressant aux objets
mobilisés par les acteurs du territoire pour son développement : pétrole, forêts, etc

%Exemple de box
\tipbox{\setlength{\parindent}{20pt}
    \noindent \textbf{Rappel~: les gisements fonciers libres}
    Les gisements fonciers libres sont les espaces cadastrés auxquels on a soustrait les principales contraintes techniques à leur mobilisation dans des opérations d'aménagements (emprises bâties, zones hydrographiques, pentes fortes, espaces pollués, etc.), ainsi que des contraintes d'usage ou de fonction (espaces dédiés à des usages publics, commerciaux, industriels, etc.).
}

\chapter{La propriété foncière communale à l'épreuve de l'austérité budgétaire et du néolibéralisme~: quelles reconfigurations~?}
\thispagestyle{plain}
\label{chap:2}
\graphicspath{PARTIE1/2_Chapitre2/images/}

Puisque la propriété foncière est un instrument important à la disposition des communes pour planifier l'aménagement, pourquoi n'est-elle pas généralisée et systématique~? Au contraire même, plusieurs indices suggèrent l'hypothèse que le foncier des communes serait en voie de réduction, dans un contexte de privatisation observé dans le monde et notamment en Europe. Comment ce contexte affecte-t-il spécifiquement la propriété foncière communale en France~? Quelles en sont les implications sur la capacité des communes à élaborer des politiques foncières et à maîtriser les opérations d'aménagement conformément à leur projet de territoire, dans l'intérêt général local~?


% Exemple d'image
\begin{figure}[!h]
    \label{fig:evol-ressources}
    \caption{Graphique d'évolution du prix des terrains à bâtir et des recettes communales}
    %\medskip
    \copyrightbox[b]{
        \centering
        \includegraphics[width=1\linewidth]{PARTIE1//2_Chapitre2//images/evol_prix_ressources.png}
        }
    {\justifying
    Réalisation~: Th. Lecourt, 2025. Sources~: DV3F v7 2022 (DGFiP-DGALN-CEREMA), Comptes consolidés des communes 2012-2021 (OFGL).\\
    
    NB~: les statistiques sont calculées à l'échelle de la France hexagonale. Les montants sont rapportés en base 100 afin de rendre comparables des ordres de grandeur différents. Les terrains à bâtir sont identifiés selon la catégorisation du CEREMA. Les transactions d'un montant inférieur à 1€ n'ont pas été prises en compte dans le calcul du prix médian au m². La version de DV3F utilisée datant de 2022, toutes les transactions réalisées en 2021 n'avaient pas été encore remontées aux services fiscaux~: le calcul du prix médian de 2021 pourrait s'en trouver affecté.}
\end{figure}


%Exemple d'encadré
\begin{tcolorbox}[title=Clés de lecture des graphiques de résultat des régressions logistiques multinomiales, colback=white!3!white, colframe=black!75!black, enhanced, breakable]
\label{sec:encadre-logit}
    Comme évoqué dans l'encadré dédié à la régression logistique du Chapitre 5, les graphiques de résultats du modèle représentent les \textit{odds-ratio} (OR) sur l'axe des abscisses. Les OR correspondent à la probabilité qu'une modalité soit observée plutôt que la modalité de référence, représentée par l'axe vertical noir. Par exemple, on peut dire que, dans l'aire d'attraction de Paris (\textit{com\_typeaav}), il est deux fois plus probable d'observer des intermédiations que des portages longs.
\end{tcolorbox}


% Conclusion de la partie 1

%Définition des entêtes
\pagestyle{fancy}
\fancyhead{}
\fancyfoot{}
\fancyhead[RE,RO]{\footnotesize \leftmark}
\fancyfoot[RE]{\footnotesize PARTIE \thepart}
\fancyfoot[LE,RO]{\footnotesize \thepage}
%
\label{chap:conclu-p1}
\addcontentsline{toc}{chapter}{Conclusion de la première partie~: questions et hypothèses de recherche}

Le rôle historique des communes, l'imaginaire municipaliste qui leur est attaché et les compétences dont elles disposent aujourd'hui, font d'elles des propriétaires fonciers essentiels à l'aménagement du territoire en France. 
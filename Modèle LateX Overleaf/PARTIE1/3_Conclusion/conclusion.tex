

Leur rôle historique, l'imaginaire municipaliste qui leur est attaché et les compétences dont elles disposent aujourd'hui, font des communes des propriétaires fonciers essentiels à l'aménagement du territoire en France. Leur propriété sert notamment à garantir une forme d'intérêt général local dans l'aménagement~: assurer des services publics, développer et gérer des infrastructures de transport et des équipements publics comme les écoles, mais aussi préserver les ressources naturelles et lutter contre l'étalement urbain et l'artificialisation des sols. Le foncier communal remplit aussi une fonction \textit{productive}~: développer l'offre de logements, de commerces, l'agriculture locale, attirer des entreprises et des emplois, organiser le développement urbain. Ces deux logiques peuvent être complémentaires, notamment dans le cas de la production de logements abordables et de logements sociaux pour favoriser l'égalité dans l'accès au logement. Dans ces objectifs, différents outils peuvent être mobilisés pour agir sur les marchés fonciers, notamment réglementaires et fiscaux, mais la propriété foncière apparaît comme une modalité d'action particulièrement efficace pour orienter des projets d'aménagement dans le sens des intérêts publics. À cette fin, les communes ont la possibilité de constituer des réserves foncières et d'intervenir plus ponctuellement sur le marché.


Les communes et les intercommunalités jouent un rôle fondamental dans les politiques d’aménagement du territoire en France. À ce titre, elles s’efforcent de s’assurer une certaine maîtrise du foncier, considéré comme une ressource territoriale stratégique. Dans une organisation sociétale caractérisée par la généralisation de la propriété privée, la propriété publique communale apparaît comme un instrument privilégié pour la mise en \oe{}uvre d’opérations d'aménagement dans l’intérêt général, telles que la production de logements, notamment sociaux, la création d’équipements publics ou encore la lutte contre l’artificialisation des sols.

L’efficacité de ce levier repose sur la préservation, voire la consolidation, du patrimoine foncier communal. Cependant, le contexte austéritaire et néolibéral contraint les collectivités locales à réduire leurs dépenses et générer de nouvelles recettes. Elles pourraient dès lors être incitées à céder une partie de leur patrimoine afin de répondre à ces impératifs financiers et leur capacité à maîtriser les orientations d'aménagement s'en trouverait limitée.

L'objectif est d'analyser la manière dont le bloc communal, en France, réagit face à ce dilemme. Elle s'appuie sur une exploitation approfondie des données cadastrales, qui consignent les données sur la propriété foncière. Ces données offrent une opportunité inédite d’appréhender empiriquement la propriété foncière communale mais soulèvent plusieurs défis méthodologiques en termes de volume et d’incertitude de l'information. Leur exploitation nécessite d’importants enrichissements afin de documenter de manière rigoureuse la diversité de la propriété foncière communale, y compris certaines formes particulières que sont les communaux et le domaine public.

Cette thèse développe ainsi une triple approche : théorique, visant à définir ce qui constitue le «~gisement foncier~» mobilisable dans des opérations d’aménagement ; une approche méthodologique pour exploiter des données cadastrales, notamment dans une perspective d’analyse spatio-temporelle des dynamiques de propriété foncière ; et une approche thématique, mettant en lumière la place de la propriété foncière communale dans le mouvement de néolibéralisation de l'aménagement.

Les résultats de ce travail conduisent à remettre en question l'hypothèse d'une disparition du foncier communal et invitent à observer plutôt sa consolidation. Des dynamiques de privatisation sont effectivement attestées, mais elles correspondent généralement à la mobilisation de foncier par les communes pour orienter les opérations d'aménagement, témoignant ainsi de leur maîtrise foncière plutôt que d'une perte de contrôle.


\textit{Mots-clefs : maîtrise foncière, propriété publique, privatisation, communes, cadastre, spatio-temporel}